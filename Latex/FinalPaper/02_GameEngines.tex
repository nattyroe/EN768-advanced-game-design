\section{Game Engines}

\begin{table}[!t]
    \renewcommand{\arraystretch}{1.3}
    \caption{Summary of Leading Game Development Engines}
    \label{tab:engineSummary}
    \centering
    \begin{tabular}{c|c|c|c|c|c} %|c}
    \hline
    \multirow{2}{*}{\bfseries Engine} & \multirow{2}{*}{\bfseries Manufacturer} & \multirow{2}{*}{\bfseries Language} & \multirow{2}{*}{\bfseries Platforms} & \multirow{2}{*}{\bfseries Tools} & \bfseries Dev \\
    & & & & & \bfseries Friendly \\ \hline \hline
    \multirow{2}{*}{Unity} & Unity & \multirow{2}{*}{C\#} & \multirow{2}{*}{\ref{sec:supportedPlatforms}} & \multirow{2}{*}{\ref{sec:tools}} & \multirow{2}{*}{Yes} \\
    & Technologies & & & & \\ \hline
    \multirow{2}{*}{Unreal} & Epic & \multirow{2}{*}{C++} & \multirow{2}{*}{\ref{sec:supportedPlatforms}} & \multirow{2}{*}{\ref{sec:tools}} & \multirow{2}{*}{Yes} \\
    & Games & & & & \\ \hline
    \multirow{3}{*}{Godot} & Godot & \multirow{2}{*}{GDScript} & \multirow{3}{*}{\ref{sec:supportedPlatforms}} & \multirow{3}{*}{\ref{sec:tools}} & \multirow{3}{*}{Yes} \\
    & Foundation & \multirow{2}{*}{C\#} & & & \\
    & (Open Source) & & & & \\ \hline \hline
    \end{tabular}
\end{table}

Development of modern games is made possible by game engines, or development platforms with specific features made for assisting in creating
games similar to current popular games. Table~\ref{tab:engineSummary} contains a summary of three leading game development engines, and some of the
features specific to each.

\subsection{Supported Platforms}\label{sec:supportedPlatforms}

Each of the game engines above support multiple target platforms, with Unity and Unreal Engine including just about every mainstream gaming
device. The joint platforms between each of these engines include iOS, Android, Windows, Linux, MacOS, and HTML5. Unity and Unreal Engine both
contain native support for compiling on leading game devices such as Microsoft Xbox, Sony Playstation, and Nintendo Switch. External tools allow
for Godot engine games to be compiled for these platforms as well, but support is not built in to the base package. Both Unity and Unreal both
additionally include support for compiling for Virtual Reality systems, such as ths Oculus Rift.

\subsection{Tools}\label{sec:tools}

Each of the game engines share a fairly similar set of tools, all aimed at providing developers the features they need to develop games from scratch.
All engines compared here contain 3D object support and game test playback for game testing. Unity and Godot additonally contain separate 2D engines
for simplifying the process for games that do not need depth. Unity and Unreal both additionally have their own asset marketplaces, containing
both paid and free assets to assist developers in the process.

\subsection{Review}\label{sec:review}

All three engines are widely used by current developers, and provide a lot of tools and great physics engines for games. Godot has not seen
widspread addoption for today's popular games, but it is user-friendly, and open source, allowing for developers to contribute to improving
the engine with features they would like to see. Unity has been fairly popular, with some well-known games coming from the engine, such as Rust
and Kerbal Space Program. However, uptake of Unity for professional development took a downturn in 2023 when they announce pricing changes that
charge per download, which has major impacts to small developers where overall game price is lower. Because of this, Unity is seen as less
friendly to indie-game developers, especially when compared to Unreal. Unreal Engine gets the highest marks here due to its widespread popularity
both for indie development, as well as development for some of the most popular AAA Games, such as Fortnite. Both Unity and Unreal are generally considered very
developer friendly, and while Unity has the largest individual developer base, Unreal has seen more uptake by existing game studios.
