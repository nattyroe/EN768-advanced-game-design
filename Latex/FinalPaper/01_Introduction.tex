
\section{Introduction}

Traditionally, Game Theory is the study of strategy and decision making in competitive scenarios, with the goal of mathematically modeling the decision making processes to determine opportune moves. In this way, Game Theory can be thought of as ``the science of strategy''~\cite{b3}. In general, the analyzed models are zero-sum competitions, where various choices are available that will lead to possible win/loss/draw scenarios for each player~\cite{b1}. Using this decision tree analysis, modeling can be developed to demonstrate likeliness of winning a competitive scenario, and how each decision affects that probability. In the context of video games, there are some games that this could directly be applied to; for example, competitive \textit{Pok\'emon} battle tournaments draw steep competition across the world and meet most of the classic game theory requirements.

Analyzing Video Game theory requires an expansion of this subject. While there is a broad overlap of subject matter, like defining rules and agent decision trees or looking at strategies for competition, the specifics are generally more complex as the competition may include more than two players,  or one player against the game's \ac{AI}. Additionally, Decision Theory applies to both traditional Game Theory and its application to Video Games. Decision Theory evaluates the set of prospects (available options) and analyzes preferences among those options for an agent in the scenario~\cite{b2}. Decision Theory can be applied to evaluate which selections are more opportune in a competitive scenario, or consider how Players may interact with the set of rules within the Video Game being developed.

Practical approaches to games involve the actual programming and creation of games, like adding code to generate game mechanics for the Player or Agent to interact with. Additional practical approaches include visuals for the game, and creation of characters, models, and environments for the game. To have a complete game, it is important to consider games in both theoretical and practical ways; to fully immerse the player, the world has to be engaging and entertaining, but it is important to have challenges for the player to overcome to create additional satisfaction. Applying Game Theory allows designers to analyze mechanics for balance, making sure that challenges are appropriate for the current player skill level, or that all options for things such as in-game loadout have appropriate positives and negatives to encourage players to engage in different but viable strategies for success in games.

What sets Video Games apart from other Physics-based computer systems, such as Simulations, is the idea that games are intended to bring some sort of satisfaction to Players in order to draw engagement, using appealing visuals and creating games systems that are enjoyable and challenging to the Player. Categorizing games by genre allows for users to assess and consider games based on similarity between similarity in rules, subject matter, and even difficulty so that Players can learn from experiences and apply these lessons to improving in skills required for certain games. For example, aim and reaction time may be critical to games in the First-Person Shooter genre, while Puzzle games may require critical or non-linear thinking to come up with solutions in the context of that game's world.

One such example of a genre is the Party Game. This genre is targeted at being played with groups of users, generally together in the same room, with relatively simple mechanics and targeted at playing multiple rounds of \ac{coop} or cooperative levels. Generally, these games are intended to be continually replayable and quick for Players to learn the basics~\cite{b4}. The specific sub-genre this paper will look at is the \ac{coop} Party Game; a good example of this type of game is \textit{Overcooked}, in which multiple players work together accomplishing tasks in a time crunch in order to prepare orders in a kitchen. The mechanics are relatively simple, but require Player coordination to succeed in the fast-paced environment. While the Players are focused on achieving their tasks, \textit{Overcooked} throws additional environmental challenges into the mix: making players avoid or extinguish fires, or rearranging the kitchen layouts to force Players to switch up their rhythms.

This paper will evaluate combining the \ac{coop} Party Game structure with another genre, \textit{Roguelikes}, to create a new, local \ac{coop} experience. The shared mechanic of Roguelike games is a repetitive gameplay loop, where Players constantly restart from the same location, but progress through the game by gaining knowledge, skills, and power-ups acquired through previous loops. These games can generally be very difficult, as they are intended to require players to iterate to progress, so players are intended to lose repeatedly. However, this repetitive nature is similar to the structure of Party Games. This game structure is intended to generate Player engagement by combining the simple mechanics requiring Player cooperation from Party Games with the skill acquisition through repetition structure of Roguelikes.
