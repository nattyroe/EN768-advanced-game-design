
\section{Introduction}
In many cases, Game Theory is the study of strategy and
decision making in competitive scenarios, with the goal of
mathematically modeling the decision making processes to
determine opportune moves. In this way, Game Theory can
be though of as ``the science of strategy''~\cite{b3}.
Traditionally, the analyzed models are zero-sum competitions,
where various choices are available
that will lead to possible win/loss/draw scenarios for each
player~\cite{b1}. Using this decision tree analysis, modeling can be
developed to demonstrate likeliness of winning a competitive
scenario, and how each decision affects that probability. In
the context of video games, there are some games that this
could directly be applied to; for example, competitive Pokemon
battle competitions draw steep competition across the world
and meet most of the classic game theory requirements. The
concepts of game theory in the context of Video Games are more
broad. 

When looking at Game theory in the context of Video Games
there is a broad overlap of  subject matter, like defining
rules and agent decision trees, or looking at strategies for
competition (with changes for more than two players,
non-zero-sum rules, etc.). Additionally, Decision Theory
applies to both traditional Game Theory and its application to
Video Games. Decision Theory evaluates the set of prospects
(available options) and analyzes preferences among those
options for an agent in the scenario~\cite{b2}. For Game
Theory, we can then evaluate which selections are more
opportune in a competitive scenario, or consider how Players
may interact with the set of rules within the Video Game
being developed.

Theory for Video Games expands on these concepts. In addition
to the application of strategy and decision making, Video
Games must be appealing to Players as the goal is to draw
engagement from the users, using appealing visuals and
creating games systems that are enjoyable and challenging to
the Player. Categorizing by genre allows for users to assess
and consider games based on similarity between similarity in
rules, subject matter, and even difficulty so that Players can
learn from experiences and apply these lessons to improving in
skills required for certain games. For example, aim and
reaction time may be critical to games in the First-Person
Shooter genre, while Puzzle games may require critical or
non-linear thinking to come up with solutions in the context
of that game’s world.

Practical approaches to games involve the actual programming
and creation of games, like adding code to generate game
mechanics for the Player or Agent to interact with.
Additional practical approaches include visuals for the game,
and creation of characters, models, and environments for the
game. To have a complete game, it is important to consider
games in both theoretical and practical ways; to fully
immerse the player, the world has to be engaging and
entertaining, but it is important to have challenges for the
player to overcome to create additional satisfaction.
Applying Game Theory allows designers to analyze mechanics
for balance, making sure that challenges are appropriate for
the current player skill level, or that all options for
things such as in-game loadout have appropriate positives
and negatives to encourage players to engage in different
but viable strategies for success in games.


