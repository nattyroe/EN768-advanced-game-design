\section{Playtest Results and Conclusion}

\subsection{Playtest Results}

In testing, two Players were each given the opportunity to test the game first
solo and then with one Developer. In both cases, each player wanted to play
the level multiple times in multiplayer, each playing a few short playthroughs
of the initial test level, until they had completed the level. This put play
time for each player in the 6 to 10 minute range; a good indication of immersion
for the short amount of content available.

In both playtests, the Player provided similar feedback. The first comment from
each Player was that the controls were not intuitive. While there were indicators
for each object in the scene that was interactable, no in-game feedback was
available to tell Players which buttons they should press, and whether buttons
needed to be held or a joystick moved. With some instruction, each Player was
able to pick play with a few minor errors. The control scheme is built to be
easily modified, so with some testing, a more intuitive control set can be
implemented. Additionally, future plans will include updates to the \ac{UI} to
indicate interaction buttons and type (press or hold).

The other main feedback received was that the game was much better in
multiplayer, and that to continue playing, more content and events need to be
added. The first part of this feedback indicates the game requires additional
balancing for single Player modes, such as event spawning based on Player count.
However, the general feedback was that the game was generally more exciting
when there was multiple events happening at once, so it may be beneficial to
focus on balancing and marketing the game as intended for multiplayer, allowing
for a more chaotic atmosphere that resonated with Players. The second part of
this feedback is a good sign for future development, with additional content
concepts described in Section~\ref{sec:conclusion}.

\subsection{Conclusion}\label{sec:conclusion}
In general, the initial game concept laid out in this paper appears to be
viable. Prototyping led to a balanced game for completing an initial level
in 3 to 4 attempts for pairs of Players, and kept Players engaged for that
time, especially in instances of highly chaotic random draws even when those
runs led to sinking. With additional playtesting and balancing, it should be
possible to reach a point where 2 or more players complete initial runs in
high paced runs. With the addition of ship upgrades, it should then be possible
to add additional, more difficult levels, creating an engaging game loop where
players do a few runs, upgrade, and do more runs in a more challenging 
environment.

The first planned development additions are a second, more challenging level
and a few availible ship upgrades, such as an additional sail or more durable
hull, that would be required for most players to complete level 2. This should
prove out the full concept of the game loop, testing whether Players want to
return to runs in order to gain points required to complete harder levels. The
second set of additions planned include new randomly spawned events, to allow
for more variability run-to-run. Events in test and conceptualization include
giant shark and squid attacks, lightening strikes, ship rolling if players do
not act to balance the ship, aquiring items from the sea with a net, and
flyaway sails that need to be reattached. These features are in allignment
with feedback received during play testing, and should give Players a more
immersive experience. It therefore seems feasible that continued development
could lead to a successful game achieving the initial goal of combining Party
style gameplay with a Roguelike loop.
